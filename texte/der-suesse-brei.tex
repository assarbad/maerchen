\section*{Der ſüße Brei}
\LTR{E}s war einmal ein armes, frommes Mädchen, das lebte mit ſeiner
Mutter allein, und ſie hatten nichts mehr zu eſſen.
Da ging das Kind hinaus in den Wald, und begegnete ihm da eine alte
Frau, die wußte ſeinen Jammer ſchon und ſchenkte ihm ein Töpfchen,
zu dem ſollt es ſagen:
\enquote{Töpfchen, koche}, ſo kochte es guten, ſüßen Hirſenbrei, und
wenn es ſagte: \enquote{Töpfchen, ſteh}, ſo hörte es wieder auf zu
kochen. Das Mädchen brachte den Topf ſeiner Mutter heim, und nun waren
ſie ihrer Armut und ihres Hungers ledig und aßen ſüßen Brei, ſooft
ſie wollten. Auf eine Zeit war das Mädchen ausgegangen, da ſprach die
Mutter \enquote{Töpfchen, koche}, da kocht es, und ſie ißt ſich ſatt;
nun will ſie, daß das Töpfchen wieder aufhören ſoll, aber ſie weiß das
Wort nicht. Alſo kocht es fort, und der Brei ſteigt über den Rand
hinaus und kocht immerzu, die Küche und das ganze Haus voll, und das
zweite Haus und dann die Straße, als wollts die ganze Welt ſatt machen,
und iſt die größte Not, und kein Mensch weiß ſich da zu helfen.
Endlich, wie nur noch ein einziges Haus übrig iſt, da kommt das Kind
heim und ſpricht nur: \enquote{Töpfchen, ſteh}, da ſteht es und hört
auf zu kochen; und wer wieder in die Stadt wollte, der mußte  ſich
durcheſſen.
