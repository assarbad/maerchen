\section*{Der Haſe und der Igel}
\LTR{D}iſſe Geſchicht is lögenhaft to vertellen, Jungens, aver wahr
is ſe doch, denn mien Grootvader, van den ick ſe hew, plegg jümmer,
wenn he ſe mie vortüerde\footnoteA{... mit Behaglichkeit vortrug},
dabi to ſeggen:
\enquote{Wahr mutt ſe doch ſien, mien Söhn, anners kunn man ſe jo
nich vertellen.} De Geſchicht hett ſick aver ſo todragen.

Et wöör an enen Sündagmorgen tor Harvesttied, jüſt as de Bookweeten
bloihde.: de Sünn wöör hellig upgaen am Hewen, de Morgenwind güng
warm över de Stoppeln, de Larken ſüngen inn'r Löhde\footnoteA{Luft},
de Immen ſumſten in den Bookweeten, un de Lühde güngen in ehren
Sündagsſtaht nah'r Kerken, un alle Kreatur wöör vergnögt, un de
Swinegel ook.

De Swinegel aver ſtünd vör siener Döhr, harr de Arm ünnerſlagen,
keek dabi in den Morgenwind hinut un quinkeleerde en lütjet Leedken
vör ſick hin, ſo good un ſo ſlecht, as nu eben am leewen
Sündagmorgen en Swinegel to ſingen pleggt. Indem he nu noch ſo half
lieſe vör ſick hin ſung, füll em up eenmal in, he künn ook wol,
mittlerwiel ſien Fro de Kinner wüsch un antröcke, en beeten in't
Feld ſpazeeren un toſehen, wie ſien Stähkröwen ſtünden. De
Stähkröwen wöören aver de nöckſten bi ſienem Huuſe, un he pleggte
mit ſiener Familie davon to eten, darüm ſahg he ſe as de ſienigen
an. Geſagt, gedahn. De Swinegel makte de Huusdöhr achter ſick to un
ſlög den Weg nah'n Felde in. He wöör noch nich gans wiet von Huuſe
un wull jüſt um den Slöbuſch\footnoteA{Schlehenbusch}, de dar vörm
Felde liggt, nah den Stähkröwenacker hinupdreien, as em de Haas
bemött, de in ähnlichen Geſchäften uutgahn wöör, nemlich um ſienen
Kohl to beſehn. As de Swinegel den Haaſen anſichtig wöör, ſo böhd he
em en fündlichen go'n Morgen. De Haas aver, de up ſiene Wies en
vörnehmer Herr was, un grausahm hochfahrtig dabi, antwoorde nicks up
den Swinegel ſienen Gruß, ſondern ſegte tom Swinegel, wobi he en
gewaltig höhniſche Miene annöhm: \enquote{Wie kummt et denn, dat du
hier all bi ſo fröhem Morgen im Felde rumlöppſt?}
\enquote{Ick gah ſpazeeren}, ſeggt de Swinegel.
\enquote{Spazeeren?} lachte de Haas, \enquote{mi ducht, du kunnſt de
Been ook wol to betern Dingen gebruuken.}
Diſſe Antword verdrööt den Swinegel ungeheuer, denn alles kunn he
verdregen, aver up ſiene Been laet he nicks komen, eben weil ſe von
Natuhr ſcheef wöören.
\enquote{Du bildſt di wohl in,} ſeggt nu de Swinegel tom Haaſen,
\enquote{as wenn du mit diene Beene mehr utrichten kunnſt?}
\enquote{Dat denk ick,} ſeggt de Haas.
\enquote{Dat kummt up'n Verſöök an,} meent de Swinegel,
\enquote{ick pareer, wenn wi in de Wett loopt, ick loop di vörbi.}
\enquote{Dat is tum Lachen, du mit diene ſcheefen Been,} ſeggt de
Haas, \enquote{aver mienetwegen mag't ſien, wenn du ſo övergroote
Luſt heſt. Wat gilt de Wett?}
\enquote{En goldne Lujedor un'n Buddel Branwien}, ſeggt de Swinegel.
\enquote{Angenahmen,} ſpröök de Haas, \enquote{ſla in, un denn
kann't gliek losgahn.}
\enquote{Nä, ſo groote Ihl hett et nich,} meen de Swinegel,
\enquote{ick bün noch gans nüchdern; eerſt will ick to Huus gahn un
en beeten fröhſtücken: inner halwen Stünd bün ick wedder hier up'n
Platz.}

Damit güng de Swinegel, denn da Haas wöör et tofreeden. Ünnerwegs
dachte de Swinegel bi ſick:
\enquote{De Haas verlett ſick up ſiene langen Been, aver ick will em
wol kriegen. He is zwar en vörnehm Herr, aver ick doch man'n dummen
Keerl, un betahlen ſall he doch.}
As nu de Swinegel to Huufe anköm, ſpröök he to ſien Fro:
\enquote{Fro, treck di gau\footnoteA{ſchnell} an, du muſtmit mi nah'n
Felde hinuut.}
\enquote{Wat givt et denn?} ſeggt ſien Fro.
\enquote{Ick hew mit'n Haaſen wett't üm'n golden Lujedor un'n Buddel
Branwien, ick will mit em inn Wett loopen, un da ſalſt du mit dabi
ſien.}
\enquote{O mien Gott, Mann,} füng nu den Swinegel ſien Fro an to
ſchreen, \enquote{büſt do nich klook, heſt du denn ganz den Verſtand
verlaaren? Wie kannſt du mit den Haaſen in de Wett loopen wollen?}
\enquote{Holt dat Muul, Wief,} ſeggt de Swinegel, \enquote{dat is
mien Saak. Reſonehr nich in Männergeſchäfte. Marſch, treck di an, un
denn kumm mit.}
Wat ſull den Swinegel ſien Fro maken? Se mußt wol folgen, ſe mugg
wollen oder nich.

As ſe nu miteenander ünnerwegs wöören, ſpröök de Swinegel to ſien
Fro:
\enquote{Nu paß up, wat ick ſeggen will. Sühſt du, up den langen
Acker dar wüll wi unſen Wettloop maken. De Haas löppt nemlich in der
eenen Föhr\footnoteA{Furche} un ick inner andern, un von baben\footnoteA{oben}
fang wi an to loopen. Nu haſt du wieder\footnoteA{weiter} nicks to
dohn, as du ſtellſt du hier unnen in de Föhr, un wenn de Haas up de
Siet ankummt, ſo röpſt du em entgegen: \enquote{Ick bün all\footnoteA{schon}
hier.}}

Damit wöören ſe bi den Acker anlangt, de Swinegel wiesde ſiener Fro
ehren Platz an un güng nu den Acker hinup. As he haben ankööm, wöör
de Haas all da.
\enquote{Kann et losgahn?} ſeggt de Haas.
\enquote{Jawol}, ſeggt de Swinegel.
\enquote{Denn man to!}
Un damit ſtellde jeder ſick in ſiene Föhr. De Haas tellde\footnoteA{zählte}:
\enquote{Hahl een, hahl twee, hahl dree}, un los güng he wie en
Stormwind den Acker hindahl\footnoteA{hinab}. De Swinegel aver lööp
ungefähr man dree Schritt, dann duhkde he ſick dahl\footnoteA{herab}
in de Föhr un bleev ruhig ſitten.

As nu de Haas in vullen Loopen ünnen am Acker ankööm, rööp em den
Swinegel ſien Fro entgegen:
\enquote{Ick bün all hier.}
De Haas ſtutzd un verwunderde ſick nich wenig: he meende nich anders,
als et wöör de Swinegel ſülvſt, de em dat torööp, denn bekanntlich
ſüht den Swinegel ſien Fro jüſt ſo uut wie ehr Mann.

De Haas aver meende:
\enquote{Dat geiht nich to mit rechten Dingen.}
He rööp:
\enquote{Nochmal geloopen, wedder üm!}
Un fort güng he wedder wie en Stormwind, dat em de Ohren am Koppe
flögen. Den Swinegel ſien Fro aver blev ruhig up ehren Platze.
As nu de Haas baben ankööm, rööp em de Swinegel entgegen:
\enquote{Ick bün all hier.}
De Haas aver, ganz uuter ſick vör Jhwer\footnoteA{Ärger}, ſchreede:
\enquote{Nochmal geloopen, wedder üm!}
\enquote{Mi nich to ſchlimm,} antwoorde de Swinegel,
\enquote{mienetwegen ſo oft  as du Luſt heſt.}
So lööp de Haas noch dreeunſöbentigmal, un de Swinegel höhl\footnoteA{hielt}
et ümmer mit em uut. Jedesmal, wenn de Haas ünnen oder baben ankööm,
ſeggten de Swinegel oder ſien Fro:
\enquote{Ick bün all hier.}
Tum veerunſöbentigſtenmal aver köm de Haas nich mehr to Ende. Midden
am Acker ſtört he tor Eerde, da Blohd flög em uut'n Halſe, un he
bleev doot up'n Platze. De Swinegel aver nöhm ſiene gewunnene Lujedor
un den Buddel Branwien, rööp ſiene Fro uut der Föhr aff, un beide
güngen vergnögt miteenanner nah Huus; un wenn ſe nich ſtorben ſünd,
lewt ſe noch.

So begev et ſick, dat up der Buxtehuder Heid de Swinegel den Haaſen
dootloopen hett, und ſied jener Tied hatt et ſick keen Haas wedder
infallen laten, mit'n Buxtehuder Swinegel in de Wett to loopen.

De Lehre aver uut diſſer Geſchicht is erſtens, dat keener un wenn he
ſick ook noch ſo vörnehm dücht, ſick ſall bikommen laten, över'n
geringen Mann ſick luſtig to maken, un wöört ook man'n Swinegel. Un
tweetens, dat et gerahden is, wenn eener freet, dat he ſick 'ne Fro
uut ſienem Stande nimmt, un de jüſt ſo uutſüht as he ſülvſt. Wer alſo
en Swinegel is, de mutt toſehn, dat ſiene Fro ook en Swinegel is, un
ſo wieder\footnoteA{weiter}.
