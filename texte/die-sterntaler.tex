\section*{Die Sterntaler}
\LTR{E}s war einmal ein kleines Mädchen, dem war Vater und Mutter
geſtorben, und es war ſo arm, daß es kein Kämmerchen mehr
hatte, darin zu wohnen, und kein Bettchen mehr, darin zu ſchlafen,
und endlich gar nichts mehr als die Kleider auf dem Leib und ein
Stückchen Brot in der Hand, das ihm ein mitleidiges Herz geſchenkt
hatte. Es war aber gut und fromm. Und weil es ſo von aller Welt
verlaſſen war, ging es im Vertrauen auf den lieben Gott hinaus ins
Feld. Da begegnete ihm ein armer Mann der ſprach: \enquote{Ach,
gib mir etwas zu eſſen ich bin ſo hungrig.} Es reichte ihm das ganze
Stückchen Brot und ſagte: \enquote{Gott ſegne es dirs!} und ging
weiter. Da kam ein Kind, das jammerte und ſprach: \enquote{Es friert
mich ſo an meinem Kopfe, ſchenk mir etwas, womit ich ihn bedecken
kann.} Da tat es ſeine Mütze ab und gab ſie ihm. Und als es noch
eine Weile gegangen war, kam wieder ein Kind und hatte kein Leibchen
an und fror, da gab es ihn ſeins; und noch weiter, da bat eins um
ein Röcklein, das gab es auch von ſich hin. Endlich gelangte es in
einen Wald, und es war ſchon dunkel geworden, da kam noch eins und
bat um ein Hemdlein, und das fromme Mädchen dachte: \enquote{Es iſt
dunkle Nacht, da ſieht dich niemand, du kannſt wohl dein Hemd
weggeben} und zog das Hemd ab und gab es auch noch hin. Und wie es
ſo ſtand und gar nichts mehr hatte, fielen auf einmal die Sterne vom
Himmel und waren lauter harte, blanke Taler: und ob es gleich ſein
Hemdlein weggegeben hattem ſo hatte es ein neues an, und das war vom
allerfeinſten Linnen. Da ſammelte es ſich die Taler hinein und war
reich für ſein Lebtag.
